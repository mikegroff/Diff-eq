\documentclass[12pt]{article}
 
\usepackage[margin=1in]{geometry}
\usepackage{amsmath,amsthm,amssymb,mathtools,amsfonts}
 
\newcommand{\N}{\mathbb{N}}
\newcommand{\R}{\mathbb{R}}
\newcommand{\Z}{\mathbb{Z}}
\newcommand{\Q}{\mathbb{Q}}
\newcommand{\defeq}{\vcentcolon=}
\newcommand{\eqdef}{=\vcentcolon}
\newcommand{\overbar}[1]{\mkern 1.5mu\overline{\mkern-1.5mu#1\mkern-1.5mu}\mkern 1.5mu}

\newenvironment{theorem}[2][Theorem]{\begin{trivlist}
\item[\hskip \labelsep {\bfseries #1}\hskip \labelsep {\bfseries #2.}]}{\end{trivlist}}
\newenvironment{lemma}[2][Lemma]{\begin{trivlist}
\item[\hskip \labelsep {\bfseries #1}\hskip \labelsep {\bfseries #2.}]}{\end{trivlist}}
\newenvironment{exercise}[2][Exercise]{\begin{trivlist}
\item[\hskip \labelsep {\bfseries #1}\hskip \labelsep {\bfseries #2.}]}{\end{trivlist}}
\newenvironment{problem}[2][Problem]{\begin{trivlist}
\item[\hskip \labelsep {\bfseries #1}\hskip \labelsep {\bfseries #2.}]}{\end{trivlist}}
\newenvironment{question}[2][Question]{\begin{trivlist}
\item[\hskip \labelsep {\bfseries #1}\hskip \labelsep {\bfseries #2.}]}{\end{trivlist}}
\newenvironment{corollary}[2][Corollary]{\begin{trivlist}
\item[\hskip \labelsep {\bfseries #1}\hskip \labelsep {\bfseries #2.}]}{\end{trivlist}}
 
\begin{document}
\title{Numerical Methods HW}
\author{Michael Groff 
}
\maketitle
\text{ }\\
\textbf{Double Pendulum Problem}

\textbf{•}\\
Derive a closed derivation for the motion of a double pendulum in vertor form.

\begin{proof}
\textbf{•}\\
We start with some definitions: let $l_1,l_2$ be the length of the rigid wires, $m_1,m_2$ the mass of the bobs, and $\theta_1, \theta_2$ the angl each of the wires makes with the defined vertical.\\
The positions of the bobs in a Cartesian plane can then be defined as:
\begin{align}
x_1 & = l_1 sin(\theta_1) \\
y_1 & = -l_1 cos(\theta_1)\\
x_2 & = l_1 sin(\theta_1)+l_2 sin(\theta_2)\\
y_2 & = -l_1 cos(\theta_1) - l_2 cos(\theta_2)
\end{align}
We then have the potential energy of the system given by:
\begin{align}
V &= m_1gy_1 + m_2gy_2 \nonumber \\
& = -(m_1+m_2)gl_1gcos(\theta_1) - m_2g l_2cos(\theta_2)
\end{align}
And the Kinetic Energy:
\begin{align}
T &= \dfrac{1}{2}m_1v^2_1 + \dfrac{1}{2}m_2v^2_2 \nonumber \\
&= \dfrac{1}{2}m_1(x^2_1+ y^2_1) + \dfrac{1}{2}m_2(x^2_2+ y^2_2) \nonumber  \\
&= \dfrac{1}{2}m_1l_1^2\dot{\theta}^2_1 + \dfrac{1}{2}(m_2l_1^2\dot{\theta}_1^2 + m_2l_2^2\dot{\theta}_2^2) + m_2l_1l_2\dot{\theta}_1 \dot{\theta}_2 cos(\theta_1 - \theta_2)
\end{align}
The Langragian of this system is then:
\begin{align}
L &\equiv T - V \nonumber \\
&= \dfrac{1}{2}(m_1+m_2)l_1^2\dot{\theta}^2_1 + \dfrac{1}{2}m_2l_2^2\dot{\theta}_2^2 + m_2l_1l_2 \dot{\theta_1}\dot{\theta}_2 cos(\theta_1 - \theta_2) \nonumber \\
&+ (m_1+m_2)gl_1gcos(\theta_1) + m_2g l_2cos(\theta_2)
\end{align}
Now for $\theta_1$:
\begin{align}
\dfrac{\partial L}{\partial \dot{\theta}_1} &= m_1l_1^2\theta_1 + m_2l_1^2\theta_1 + m_2l_1l_2\theta_2cos(\theta_1 - \theta_2)  \\
\dfrac{d}{dt}(\dfrac{\partial L}{\partial \dot{\theta}_1}) &= (m_1+m_2)l_1^2\ddot{\theta_1} + m_2l_1l_2\ddot{\theta}_2 cos(\theta_1 - \theta_2) - m_2l_1l_2\dot{\theta}_2sin(\theta_1-\theta_2)(\dot{\theta_1}-\dot{\theta_2}) \\
\dfrac{\partial L}{\partial \theta_1} &= -l_1g(m_1+m_2)sin(\theta_1) - m_2l_1l_2\dot{\theta}_1 \dot{\theta}_2 sin(\theta_1-\theta_2)
\end{align}
Then the Euler Lagrange equation for $\theta_1$ is $\dfrac{d}{dt}(\dfrac{\partial L}{\partial \dot{\theta}_1}) - \dfrac{\partial L}{\partial \theta_1} = 0$ which is equivalent to:
\[(m_1+m_2)l_1^2\ddot{\theta}_1 + m_2l_1l_2\ddot{\theta}_2cos(\theta_1-\theta_2)+m_2l_1l_2\dot{\theta}_2^2sin(\theta_1-\theta_2) + l_1g(m_1+m_2)sin(\theta_1) = 0\]
We can divide through by $l_1$ yielding:
\[(m_1+m_2)l_1\ddot{\theta}_1 + m_2l_2\ddot{\theta}_2cos(\theta_1-\theta_2)+m_2l_2\dot{\theta}_2^2sin(\theta_1-\theta_2) + g(m_1+m_2)sin(\theta_1) = 0\]
Repeating this process for $\theta_2$:
\begin{align}
\dfrac{\partial L}{\partial \dot{\theta}_2} &=  m_2l_2^2\theta_1 + m_2l_1l_2\theta_2cos(\theta_1 - \theta_2)  \\
\dfrac{d}{dt}(\dfrac{\partial L}{\partial \dot{\theta}_2}) &= (m_1+m_2)l_2^2\ddot{\theta_1} + m_2l_1l_2\ddot{\theta}_2 cos(\theta_1 - \theta_2) - m_2l_1l_2\dot{\theta}_2sin(\theta_1-\theta_2)(\dot{\theta_1}-\dot{\theta_2}) \\
\dfrac{\partial L}{\partial \theta_2} &= m_2l_1l_2\dot{\theta}_1 \dot{\theta}_2 sin(\theta_1-\theta_2) - l_2gm_2sin(\theta_2)
\end{align}
Then the Euler Lagrange equation for $\theta_2$ is:
\[m_2l_2^2\ddot{\theta}_2 + m_2l_1l_2\ddot{\theta}_2 cos(\theta_1 - \theta_2) - m_2l_1l_2\dot{\theta}_2^2sin(\theta_1-\theta_2) + l_2m_2gsin(\theta_2) = 0\]
Dividing through by $l_2$ and $m_2$ this time yields:
\[l_2\ddot{\theta}_2 + l_1\ddot{\theta}_2 cos(\theta_1 - \theta_2) - l_1l_2\dot{\theta}_2^2sin(\theta_1-\theta_2) + gsin(\theta_2) = 0\]
Now we can substitute variables $x_1,x_2,x_3,x_4$ in for $\theta_1,\theta_2, \dot{\theta}_1, \dot{\theta}_2$ respectively and letting $\Delta \theta = \theta_1-\theta_2$ and $M = m_1+m_2$ to obtain system of equations:
\begin{align*}
Ml_1^2\dot{x_3} + m_2l_2\dot{x_4}cos(\Delta \theta)-m_2l_2x_4^2sin(\theta_1-\theta_2) + Mgsin(x_1) &= 0 \\
l_2\dot{x_4} + l_1\dot{x}_2 cos(\Delta \theta) - l_1x_3^2sin(\Delta \theta) + gsin(x_2) &= 0
\end{align*}
Then we can use the substitutions $C = cos(\Delta \theta), S = sin(\Delta \theta)$ and turn this into a matrix of equations:
\[ \left( \begin{array}{cc}
Ml_1 & m_2l_2C \\
l_1C & l_2
\end{array} \right)
%
\left( \begin{array}{c}
\dot{x}_3  \\
\dot{x}_4 
\end{array} \right)
=
\left( \begin{array}{c}
m_2l_2x_4^2S-Mgsin(x_1) \\
-l_1x_3^2S-gsin(x_2)
\end{array} \right)
\]
Multiplying both sides by the inverse of the first matrix with determinant:
\[D = Ml_1l_2 - m_2l_1l_2C^2  = l_1l_2M(1-\dfrac{m_2}{M}C^2) > 0\]
We obtain the system: 
\[\left( \begin{array}{c}
\dot{x}_3  \\
\dot{x}_4 
\end{array} \right)
=
\dfrac{1}{D}
\left( \begin{array}{cc}
l_2 & -m_2l_2C  \\
-l_1C & Ml_1
\end{array} \right)
\left( \begin{array}{c}
m_2l_2x_4^2S-Mgsin(x_1) \\
-l_1x_3^2S-gsin(x_2)
\end{array} \right)
\equiv
\left( \begin{array}{c}
F_3 \\
F_4
\end{array} \right)
\]
Thus our total system is:
\[
\mathbb{X} = 
\left( \begin{array}{c}
\dot{x}_1 \\
\dot{x}_2 \\
\dot{x}_3 \\
\dot{x}_4
\end{array} \right)
=
\left( \begin{array}{c}
x_3 \\
x_4 \\
F_3 \\
F_4
\end{array} \right)
; \textbf{  }
\mathbb{X}_0 = 
\left( \begin{array}{c}
x_1 \\
x_2 \\
x_3 \\
x_4
\end{array} \right)
(0)
\]
Where 
\[\mathbb{X}' = F(t,\mathbb{X}); \textbf{ } \mathbb{X}_0 = \mathbb{X}(0) \]




\end{proof}

\end{document}


